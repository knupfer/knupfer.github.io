% Created 2013-10-28 Mon 12:53
\documentclass[11pt]{article}
\usepackage[utf8]{inputenc}
\usepackage[T1]{fontenc}
\usepackage{fixltx2e}
\usepackage{graphicx}
\usepackage{longtable}
\usepackage{float}
\usepackage{wrapfig}
\usepackage{rotating}
\usepackage[normalem]{ulem}
\usepackage{amsmath}
\usepackage{textcomp}
\usepackage{marvosym}
\usepackage{wasysym}
\usepackage{amssymb}
\usepackage{hyperref}
\tolerance=1000
\author{FK}
\date{\today}
\title{about}
\hypersetup{
  pdfkeywords={},
  pdfsubject={},
  pdfcreator={Emacs 24.2.1 (Org mode 8.2)}}
\begin{document}

\maketitle
\tableofcontents


\section{Hobbies}
\label{sec-1}
\subsection{Kultur}
\label{sec-1-1}
\subsubsection{Kochen}
\label{sec-1-1-1}
\subsubsection{Sprachen}
\label{sec-1-1-2}
\subsubsection{Reisen}
\label{sec-1-1-3}
\subsubsection{Typographie}
\label{sec-1-1-4}

\subsection{Technik}
\label{sec-1-2}
\subsubsection{Tablets}
\label{sec-1-2-1}
\subsubsection{Elektroautos}
\label{sec-1-2-2}
\subsubsection{Energieproduktion}
\label{sec-1-2-3}

\section{Musiker}
\label{sec-2}
\subsection{Orgel}
\label{sec-2-1}
\subsubsection{Literatur}
\label{sec-2-1-1}
\subsubsection{Improvisation}
\label{sec-2-1-2}

\subsection{Gesang}
\label{sec-2-2}
\subsubsection{Bass}
\label{sec-2-2-1}
\subsubsection{Altus}
\label{sec-2-2-2}

\section{Curriculum Vitae}
\label{sec-3}
\subsection{Persönliche Daten}
\label{sec-3-1}
\begin{center}
\begin{tabular}{ll}
\hline
Name & Florian Klaus Knupfer\\
Geburtsdatum & 18.10.1987\\
Geburtsort & Aalen\\
Eltern & Klaus Knupfer Ursula Knupfer, geb. Schmid\\
Familienstand & Verheiratet\\
\hline
\end{tabular}
\end{center}
\subsection{Ausbildung}
\label{sec-3-2}
\begin{center}
\begin{tabular}{rl}
\hline
1994–1998 & Grundschule Parkschule Essingen\\
1998–2007 & Theodor-Heuss-Gymnasium Aalen, Preis der Schule und Preis der DPG (Deutsche Physikalische Gesellschaft)\\
2007–2008 & C-Kurs Rottenburg am Neckar\\
seit 2008 & Doppelstudium Schulmusik und Kirchenmusik an der Hochschule für Musik und Darstellende Kunst Stuttgart\\
seit 2010 & Physikstudium an der Universität Vaihingen\\
\hline
\end{tabular}
\end{center}
\subsection{Sonstige Tätigkeiten}
\label{sec-3-3}
\begin{center}
\begin{tabular}{rl}
\hline
2008–2012 & Kirchenchorleiter und Organist der Marienwallfahrtskirche alen-Unterkochen\\
2008 & Studienreise nach Italien um die dortige Orgellandschaft kennenzulernen\\
2010 & Studienreise nach Toulouse, Dolmetscher für die Studenten um sich in die frz. Kultur integrieren zu können\\
2010 & Mehrere Uraufführungen eigener Kompositionen durch den Aalener Oratorienchor\\
2010 & Stipendiat des »Europäischen Chorforums für junge Komponisten« Ochsenhausen, in diesem Rahmen Aufführung eigener Werke durch das Orpheus Vokalensemble\\
2011 & Abschlussprüfung in Chorleitung, Orchesterleitung und in Musiktheorie\\
2011 & Praktikum beim Landesgymnasium für Hochbegabte Schwäbisch Gmünd\\
seit 2012 & Mitglied im MinD-Hochschulnetzwerk www.mhn.mensa.de\\
2012 & Träger des Deutschlandstipendiums\\
2012 & Stipendiat des DAAD go east! Programms an der Staatlichen Landesuniversität Russlands in Moskau\\
seit Jan 2013 & Leiter der Musikschule Obertürkheim Stuttgart www.musikschule-obertuerkheim.de\\
Feb 2013 & Stipendiat der Karin Abt-Straubinger-Stiftung für eine Sprachschule an der technischen Staatsuniversität in Irkutsk, Sibirien\\
seit Feb 2013 & Mitglied im Doktorandennetzwerk des AMPF (Arbeitskreis Musikpädagogische Forschung)\\
Aug 2013 & Konzert im Dom zu Schweidnitz (Polen)\\
Aug 2013 & Staatsexamensarbeit: »Abgrenzung von Musikalität und musikalischer Kompetenz – Eine empirische Untersuchung von Over- und Underachievement bei Sechstklässlern«; Note 1,0\\
\hline
\end{tabular}
\end{center}
\subsection{Sprachen}
\label{sec-3-4}
\begin{center}
\begin{tabular}{ll}
\hline
Englisch & Sehr gut in Wort und Schrift\\
Spanisch & Sehr gut in Wort und Schrift\\
Französisch & Sehr gut in Wort, gut in Schrift\\
Russisch & Grundlegende Kenntnisse (A2)\\
\hline
\end{tabular}
\end{center}
\subsection{Computer}
\label{sec-3-5}
\begin{center}
\begin{tabular}{ll}
\hline
\LaTeX{} & Textpublikationserfahrungen\\
Python & Grundlegende Programmierkenntnisse\\
Org-mode & Projektplanungs- und Protokollierungserfahrung\\
ILIAS und Moodle & Serverinstallation und -benutzung\\
SPSS & Profunde Statistikkenntnis\\
Lilypond & Auszeichnungssprache für professionellen Notensatz\\
\hline
\end{tabular}
\end{center}
% Emacs 24.2.1 (Org mode 8.2)
\end{document}
